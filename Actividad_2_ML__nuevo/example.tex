% Template:     Informe LaTeX
% Documento:    Archivo de ejemplo
% Versión:      8.1.7 (24/07/2022)
% Codificación: UTF-8
%
% Autor: Pablo Pizarro R.
%        pablo@ppizarror.com
%
% Manual template: [https://latex.ppizarror.com/informe]
% Licencia MIT:    [https://opensource.org/licenses/MIT]

% ------------------------------------------------------------------------------
% NUEVA SECCIÓN
% ------------------------------------------------------------------------------
% Las secciones se inician con \section, si se quiere una sección sin número se
% pueden usar las funciones \sectionanum (sección sin número) o la función
% \sectionanumnoi para crear el mismo título sin numerar y sin aparecer en el índice
\section{Plot de Datos Tomados - General}

\begin{figure}
    \centering
    \includegraphics[width=0.9\linewidth]{cccc.png}
    \caption{Enter Caption}
    \label{fig:placeholder}
\end{figure}

\section{KERNEL RIDGE - Curva de Validación de los datos}

\begin{figure}[H]
    \centering
    \includegraphics[width=0.75\linewidth]{entreamiento-humedad.png}
    \caption{Curva de Validación del Modelo para Humedad}
    \label{fig:val_humedad}
\end{figure}

\noindent Mejor gamma para Humedad: 0.100000


\begin{figure}[H]
    \centering
    \includegraphics[width=0.5\linewidth]{curav de validacion temperatura.png}
    \caption{Curva de Validación del Modelo para Temperatura}
    \label{fig:placeholder}
\end{figure}


\noindent Mejor gamma para Temperatura: 0.000001

\section{KERNEL RIDGE - Gráfica de Pariedad de los datos}

\begin{figure}[H]
    \centering
    \includegraphics[width=0.5\linewidth]{grarfica de pariedad.png}
    \caption{Gráfica de pariedad de la Humedad real vs la predicha}
    \label{fig:pariedad_humedad}
\end{figure}

\noindent \textbf{ENTRENANDO MODELO PARA HUMEDAD}\\
MSE Entrenamiento (Humedad): 4.7960\\
MSE Validación (Humedad): 5.7546\\
R\textsuperscript{2} Validación (Humedad): 0.5157\\

\begin{figure}[H]
    \centering
    \includegraphics[width=0.5\linewidth]{pariedad temperatura.png}
    \caption{Gráfica de pariedad de la Temperatura real vs la predicha}
    \label{fig:placeholder}
\end{figure}

\noindent \textbf{ENTRENANDO MODELO PARA TEMPERATURA}\\
MSE Entrenamiento (Temperatura): 1.1788\\
MSE Validación (Temperatura): 1.4142\\
R\textsuperscript{2} Validación (Temperatura): -1.6287\\

\section{KERNEL RIDGE - Curvas de aprendizaje}

\begin{figure}[H]
    \centering
    \includegraphics[width=0.75\linewidth]{curva de aprendizaje.png}
    \caption{Curvas de aprendizaje de la Humedad y Temperatura}
    \label{fig:curvas_aprendizaje}
\end{figure}

\subsection{Análisis de las curvas de aprendizaje}

\noindent \textbf{HUMEDAD:}\\
- MSE Final Entrenamiento: 4.5065\\
- MSE Final Validación: 7.8341\\
- Diagnóstico: SUBAJUSTE\\

\noindent \textbf{TEMPERATURA:}\\
- MSE Final Entrenamiento: 1.1513\\
- MSE Final Validación: 2.2288\\
- Diagnóstico: SUBAJUSTE\\

\noindent \textbf{INTERPRETACIÓN:}\\
• Si las curvas CONVERGEN: Modelo generaliza bien\\
• Si hay BRECHA grande: Posible sobreajuste\\
• Si ambas son ALTAS: Posible subajuste\\
• Áreas sombreadas: Variabilidad (1 desviación estándar)\\

\section{KERNEL RIDGE - Compromiso Sesgo-Varianza}

\begin{figure}[H]
    \centering
    \includegraphics[width=0.75\linewidth]{descargar.png}
    \caption{Compromiso Sesgo-Varianza de la Temperatura}
    \label{fig:sesgo_varianza_temp}
\end{figure}

\noindent Mejor gamma encontrado: 0.0129\\
MSE validación mínimo: 23.7237\\
MSE entrenamiento correspondiente: 21.4689\\

\begin{figure}[H]
    \centering
    \includegraphics[width=0.75\linewidth]{compromiso ses-variance humedad.png}
    \caption{Compromiso Sesgo-Varianza de la Humedad}
    \label{fig:sesgo_varianza_humedad}
\end{figure}



\noindent \textbf{DIAGNÓSTICO}\\
• Mejor gamma: 0.138950\\
• MSE validación mínimo: 7.1422\\
• MSE entrenamiento correspondiente: 3.6402\\

\noindent \textbf{Diagnóstico del problema:} SOBREAJUSTE\\
El modelo se ajusta demasiado a los datos de entrenamiento\\

\noindent \textbf{Comportamiento de gamma:}\\
• Gamma pequeño (<0.0052): Modelo muy simple (alto sesgo)\\
• Gamma óptimo (~0.1389): Balance ideal\\
• Gamma grande (>19.31): Modelo muy complejo (alta varianza)\\



\section{KERNEL RIDGE - Temperatura}

\begin{figure}[H]
    \centering
    \includegraphics[width=0.5\linewidth]{2.png}
    \caption{CPredicciones del modelo Kernel Ridge}
\end{figure}


\begin{figure}[H]
    \centering
    \includegraphics[width=0.5\linewidth]{3.png}
    \caption{Comparación visual entre valores reales y predichos}
\end{figure}

\begin{figure}[H]
    \centering
    \includegraphics[width=0.5\linewidth]{eeeeeee.png}
    \caption{Distribución de errores del modelo}
\end{figure}

\begin{figure}[H]
    \centering
    \includegraphics[width=0.5\linewidth]{5.png}
    \caption{}
\end{figure}

\begin{table}[H]
    \centering
    \caption{Resultados del modelo Kernel Ridge}
    \begin{tabular}{@{}lc@{}}
        \toprule
        \textbf{Métrica} & \textbf{Kernel Ridge} \\
        \midrule
        MSE (Test) & 0.0330 \\
        R\textsuperscript{2} (Test) & 0.8488 \\
        Error Promedio (°C) & 0.0436 \\
        Error Máximo (°C) & 0.7905 \\
        Error Mínimo (°C) & 0.0003 \\
        Desviación Estándar del Error & 0.0748 \\
        \bottomrule
    \end{tabular}
\end{table}


%%%%%%%%%%%%%%
\section{KERNEL RIDGE - Humedad}

\begin{figure}[H]
    \centering
    \includegraphics[width=0.75\linewidth]{naaaa.png}
    \caption{VISUALIZACIÓN COMPLETA CON TODOS LOS DATOS}
    \label{fig:placeholder}
\end{figure}

\begin{figure}[H]
    \centering
    \includegraphics[width=0.75\linewidth]{nddddssd.png}
    \caption{Mejor modelo tomado, regresion lineal, comparacion de humedad predicha vs humedad real}
    \label{fig:placeholder}
\end{figure}

\begin{figure}[H]
    \centering
    \includegraphics[width=0.75\linewidth]{xxxx.png}
    \caption{Evolución del error en el tiempo}
    \label{fig:placeholder}
\end{figure}



\noindent ESTADÍSTICAS DETALLADAS\\
R² en prueba: 0.9737\\
Error absoluto promedio: 0.0692%\\
Error máximo: 0.3572%\\
Error mínimo: 0.0002%\\
Desviación estándar del error: 0.0625%\\

% ======================= REGRESIÓN LINEAL =======================

\section{Regresión Lineal - Temperatura}

% Curva de validación
\begin{figure}[H]
    \centering
    \includegraphics[width=0.75\linewidth]{img/val_lineal_temp.png}
    \caption{Curva de Validación del Modelo Lineal para Temperatura}
    \label{fig:val_lineal_temp}
\end{figure}

% Espacio para comentar el mejor hiperparámetro
\noindent Mejor configuración: ...\\

% Curva de paridad
\begin{figure}[H]
    \centering
    \includegraphics[width=0.5\linewidth]{img/paridad_lineal_temp.png}
    \caption{Curva de paridad de la Temperatura real vs la predicha (Lineal)}
    \label{fig:paridad_lineal_temp}
\end{figure}

\noindent \textbf{ENTRENANDO MODELO LINEAL PARA TEMPERATURA}\\
MSE Entrenamiento: ...\\
MSE Validación: ...\\
R\textsuperscript{2} Validación: ...\\

% Curva de aprendizaje
\begin{figure}[H]
    \centering
    \includegraphics[width=0.75\linewidth]{img/aprendizaje_lineal_temp.png}
    \caption{Curva de aprendizaje para Temperatura (Lineal)}
    \label{fig:apr_lineal_temp}
\end{figure}

% Espacio para análisis
\noindent Comentario/análisis: ...\\

% ======================= REGRESIÓN LINEAL - HUMEDAD =======================

\section{Regresión Lineal - Humedad}

% Curva de validación
\begin{figure}[H]
    \centering
    \includegraphics[width=0.75\linewidth]{img/val_lineal_hum.png}
    \caption{Curva de Validación del Modelo Lineal para Humedad}
    \label{fig:val_lineal_hum}
\end{figure}

\noindent Mejor configuración: ...\\

% Curva de paridad
\begin{figure}[H]
    \centering
    \includegraphics[width=0.5\linewidth]{img/paridad_lineal_hum.png}
    \caption{Curva de paridad de la Humedad real vs la predicha (Lineal)}
    \label{fig:paridad_lineal_hum}
\end{figure}

\noindent \textbf{ENTRENANDO MODELO LINEAL PARA HUMEDAD}\\
MSE Entrenamiento: ...\\
MSE Validación: ...\\
R\textsuperscript{2} Validación: ...\\

% Curva de aprendizaje
\begin{figure}[H]
    \centering
    \includegraphics[width=0.75\linewidth]{img/aprendizaje_lineal_hum.png}
    \caption{Curva de aprendizaje para Humedad (Lineal)}
    \label{fig:apr_lineal_hum}
\end{figure}

\noindent Comentario/análisis: ...\\

% ======================= REGRESIÓN POLINÓMICA =======================

\section{Regresión Polinómica - Temperatura}

% Curva de validación
\begin{figure}[H]
    \centering
    \includegraphics[width=0.75\linewidth]{img/val_poly_temp.png}
    \caption{Curva de Validación del Modelo Polinómico para Temperatura}
    \label{fig:val_poly_temp}
\end{figure}

\noindent Mejor grado: ...\\

% Curva de paridad
\begin{figure}[H]
    \centering
    \includegraphics[width=0.5\linewidth]{img/paridad_poly_temp.png}
    \caption{Curva de paridad de la Temperatura real vs la predicha (Polinómica)}
    \label{fig:paridad_poly_temp}
\end{figure}

\noindent \textbf{ENTRENANDO MODELO POLINÓMICO PARA TEMPERATURA}\\
MSE Entrenamiento: ...\\
MSE Validación: ...\\
R\textsuperscript{2} Validación: ...\\

% Curva de aprendizaje
\begin{figure}[H]
    \centering
    \includegraphics[width=0.75\linewidth]{img/aprendizaje_poly_temp.png}
    \caption{Curva de aprendizaje para Temperatura (Polinómica)}
    \label{fig:apr_poly_temp}
\end{figure}

\noindent Comentario/análisis: ...\\

% ======================= REGRESIÓN POLINÓMICA - HUMEDAD =======================

\section{Regresión Polinómica - Humedad}

% Curva de validación
\begin{figure}[H]
    \centering
    \includegraphics[width=0.75\linewidth]{img/val_poly_hum.png}
    \caption{Curva de Validación del Modelo Polinómico para Humedad}
    \label{fig:val_poly_hum}
\end{figure}

\noindent Mejor grado: ...\\

% Curva de paridad
\begin{figure}[H]
    \centering
    \includegraphics[width=0.5\linewidth]{img/paridad_poly_hum.png}
    \caption{Curva de paridad de la Humedad real vs la predicha (Polinómica)}
    \label{fig:paridad_poly_hum}
\end{figure}

\noindent \textbf{ENTRENANDO MODELO POLINÓMICO PARA HUMEDAD}\\
MSE Entrenamiento: ...\\
MSE Validación: ...\\
R\textsuperscript{2} Validación: ...\\

% Curva de aprendizaje
\begin{figure}[H]
    \centering
    \includegraphics[width=0.75\linewidth]{img/aprendizaje_poly_hum.png}
    \caption{Curva de aprendizaje para Humedad (Polinómica)}
    \label{fig:apr_poly_hum}
\end{figure}

\noindent Comentario/análisis: ...\\

% ======================= RIDGE =======================

\section{Ridge - Temperatura}

% Curva de validación
\begin{figure}[H]
    \centering
    \includegraphics[width=0.75\linewidth]{img/val_ridge_temp.png}
    \caption{Curva de Validación del Modelo Ridge para Temperatura}
    \label{fig:val_ridge_temp}
\end{figure}

\noindent Mejor alpha: ...\\

% Curva de paridad
\begin{figure}[H]
    \centering
    \includegraphics[width=0.5\linewidth]{img/paridad_ridge_temp.png}
    \caption{Curva de paridad de la Temperatura real vs la predicha (Ridge)}
    \label{fig:paridad_ridge_temp}
\end{figure}

\noindent \textbf{ENTRENANDO MODELO RIDGE PARA TEMPERATURA}\\
MSE Entrenamiento: ...\\
MSE Validación: ...\\
R\textsuperscript{2} Validación: ...\\

% Curva de aprendizaje
\begin{figure}[H]
    \centering
    \includegraphics[width=0.75\linewidth]{img/aprendizaje_ridge_temp.png}
    \caption{Curva de aprendizaje para Temperatura (Ridge)}
    \label{fig:apr_ridge_temp}
\end{figure}

\noindent Comentario/análisis: ...\\

% ======================= RIDGE - HUMEDAD =======================

\section{Ridge - Humedad}

% Curva de validación
\begin{figure}[H]
    \centering
    \includegraphics[width=0.75\linewidth]{img/val_ridge_hum.png}
    \caption{Curva de Validación del Modelo Ridge para Humedad}
    \label{fig:val_ridge_hum}
\end{figure}

\noindent Mejor alpha: ...\\

% Curva de paridad
\begin{figure}[H]
    \centering
    \includegraphics[width=0.5\linewidth]{img/paridad_ridge_hum.png}
    \caption{Curva de paridad de la Humedad real vs la predicha (Ridge)}
    \label{fig:paridad_ridge_hum}
\end{figure}

\noindent \textbf{ENTRENANDO MODELO RIDGE PARA HUMEDAD}\\
MSE Entrenamiento: ...\\
MSE Validación: ...\\
R\textsuperscript{2} Validación: ...\\

% Curva de aprendizaje
\begin{figure}[H]
    \centering
    \includegraphics[width=0.75\linewidth]{img/aprendizaje_ridge_hum.png}
    \caption{Curva de aprendizaje para Humedad (Ridge)}
    \label{fig:apr_ridge_hum}
\end{figure}

\noindent Comentario/análisis: ...\\

% ======================= LASSO =======================

\section{Lasso - Temperatura}

% Curva de validación
\begin{figure}[H]
    \centering
    \includegraphics[width=0.75\linewidth]{img/val_lasso_temp.png}
    \caption{Curva de Validación del Modelo Lasso para Temperatura}
    \label{fig:val_lasso_temp}
\end{figure}

\noindent Mejor alpha: ...\\

% Curva de paridad
\begin{figure}[H]
    \centering
    \includegraphics[width=0.5\linewidth]{img/paridad_lasso_temp.png}
    \caption{Curva de paridad de la Temperatura real vs la predicha (Lasso)}
    \label{fig:paridad_lasso_temp}
\end{figure}

\noindent \textbf{ENTRENANDO MODELO LASSO PARA TEMPERATURA}\\
MSE Entrenamiento: ...\\
MSE Validación: ...\\
R\textsuperscript{2} Validación: ...\\

% Curva de aprendizaje
\begin{figure}[H]
    \centering
    \includegraphics[width=0.75\linewidth]{img/aprendizaje_lasso_temp.png}
    \caption{Curva de aprendizaje para Temperatura (Lasso)}
    \label{fig:apr_lasso_temp}
\end{figure}

\noindent Comentario/análisis: ...\\

% ======================= LASSO - HUMEDAD =======================

\section{Lasso - Humedad}

% Curva de validación
\begin{figure}[H]
    \centering
    \includegraphics[width=0.75\linewidth]{img/val_lasso_hum.png}
    \caption{Curva de Validación del Modelo Lasso para Humedad}
    \label{fig:val_lasso_hum}
\end{figure}

\noindent Mejor alpha: ...\\

% Curva de paridad
\begin{figure}[H]
    \centering
    \includegraphics[width=0.5\linewidth]{img/paridad_lasso_hum.png}
    \caption{Curva de paridad de la Humedad real vs la predicha (Lasso)}
    \label{fig:paridad_lasso_hum}
\end{figure}

\noindent \textbf{ENTRENANDO MODELO LASSO PARA HUMEDAD}\\
MSE Entrenamiento: ...\\
MSE Validación: ...\\
R\textsuperscript{2} Validación: ...\\

% Curva de aprendizaje
\begin{figure}[H]
    \centering
    \includegraphics[width=0.75\linewidth]{img/aprendizaje_lasso_hum.png}
    \caption{Curva de aprendizaje para Humedad (Lasso)}
    \label{fig:apr_lasso_hum}
\end{figure}

\noindent Comentario/análisis: ...\\

% ======================= ELASTICNET =======================

\section{ElasticNet - Temperatura}

% Curva de validación
\begin{figure}[H]
    \centering
    \includegraphics[width=0.75\linewidth]{img/val_enet_temp.png}
    \caption{Curva de Validación del Modelo ElasticNet para Temperatura}
    \label{fig:val_enet_temp}
\end{figure}

\noindent Mejor alpha/l1	extunderscore ratio: ...\\

% Curva de paridad
\begin{figure}[H]
    \centering
    \includegraphics[width=0.5\linewidth]{img/paridad_enet_temp.png}
    \caption{Curva de paridad de la Temperatura real vs la predicha (ElasticNet)}
    \label{fig:paridad_enet_temp}
\end{figure}

\noindent \textbf{ENTRENANDO MODELO ELASTICNET PARA TEMPERATURA}\\
MSE Entrenamiento: ...\\
MSE Validación: ...\\
R\textsuperscript{2} Validación: ...\\

% Curva de aprendizaje
\begin{figure}[H]
    \centering
    \includegraphics[width=0.75\linewidth]{img/aprendizaje_enet_temp.png}
    \caption{Curva de aprendizaje para Temperatura (ElasticNet)}
    \label{fig:apr_enet_temp}
\end{figure}

\noindent Comentario/análisis: ...\\

% ======================= ELASTICNET - HUMEDAD =======================

\section{ElasticNet - Humedad}

% Curva de validación
\begin{figure}[H]
    \centering
    \includegraphics[width=0.75\linewidth]{img/val_enet_hum.png}
    \caption{Curva de Validación del Modelo ElasticNet para Humedad}
    \label{fig:val_enet_hum}
\end{figure}

\noindent Mejor alpha/l1	extunderscore ratio: ...\\

% Curva de paridad
\begin{figure}[H]
    \centering
    \includegraphics[width=0.5\linewidth]{img/paridad_enet_hum.png}
    \caption{Curva de paridad de la Humedad real vs la predicha (ElasticNet)}
    \label{fig:paridad_enet_hum}
\end{figure}

\noindent \textbf{ENTRENANDO MODELO ELASTICNET PARA HUMEDAD}\\
MSE Entrenamiento: ...\\
MSE Validación: ...\\
R\textsuperscript{2} Validación: ...\\

% Curva de aprendizaje
\begin{figure}[H]
    \centering
    \includegraphics[width=0.75\linewidth]{img/aprendizaje_enet_hum.png}
    \caption{Curva de aprendizaje para Humedad (ElasticNet)}
    \label{fig:apr_enet_hum}
\end{figure}

\noindent Comentario/análisis: ...\\


