% Template:     Informe LaTeX
% Documento:    Archivo de ejemplo
% Versión:      8.1.7 (24/07/2022)
% Codificación: UTF-8
%
% Autor: Pablo Pizarro R.
%        pablo@ppizarror.com
%
% Manual template: [https://latex.ppizarror.com/informe]
% Licencia MIT:    [https://opensource.org/licenses/MIT]

% ------------------------------------------------------------------------------
% NUEVA SECCIÓN
% ------------------------------------------------------------------------------
% Las secciones se inician con \section, si se quiere una sección sin número se
% pueden usar las funciones \sectionanum (sección sin número) o la función
% \sectionanumnoi para crear el mismo título sin numerar y sin aparecer en el índice
\section{Plot de Datos Tomados - General}

\begin{figure}
    \centering
    \includegraphics[width=0.9\linewidth]{cccc.png}
    \caption{Datos generales: Temperatura y Humedad}
    \label{fig:toma_de_datos_generales}
\end{figure}

\textbf{Comentario sobre el mostrado de datos:} 
El gráfico general de los datos recolectados permite observar el comportamiento de la temperatura y la humedad en el tiempo, mostrando su variabilidad y tendencia general. Este paso fue fundamental para identificar patrones iniciales, posibles valores atípicos y la correlación entre ambas variables antes de aplicar los diferentes modelos predictivos. La visualización confirma que la temperatura y la humedad tienen comportamientos inversamente proporcionales en ciertos intervalos, lo que respalda la coherencia de los datos obtenidos con el sensor y justifica su uso en los modelos posteriores.



\section{KERNEL RIDGE - Curva de Validación de los datos}

\begin{figure}[H]
    \centering
    \includegraphics[width=0.75\linewidth]{entreamiento-humedad.png}
    \caption{Curva de Validación del Modelo para Humedad}
    \label{fig:val_humedad}
\end{figure}

\noindent Mejor gamma para Humedad: 0.100000


\begin{figure}[H]
    \centering
    \includegraphics[width=0.5\linewidth]{curva kernel temp.png}
    \caption{Curva de Validación del Modelo para Temperatura}
    \label{fig:curva_de_validacion_temp_ker_ridge}
\end{figure}


\noindent Mejor gamma para Temperatura: 0.000001

\section{KERNEL RIDGE - Gráfica de Pariedad de los datos}

\begin{figure}[H]
    \centering
    \includegraphics[width=0.5\linewidth]{grarfica de pariedad.png}
    \caption{Gráfica de pariedad de la Humedad real vs la predicha}
    \label{fig:pariedad_humedad}
\end{figure}

\noindent \textbf{ENTRENANDO MODELO PARA HUMEDAD}\\
MSE Entrenamiento (Humedad): 4.7960\\
MSE Validación (Humedad): 5.7546\\
R\textsuperscript{2} Validación (Humedad): 0.5157\\

\begin{figure}[H]
    \centering
    \includegraphics[width=0.5\linewidth]{pariedad temperatura.png}
    \caption{Gráfica de pariedad de la Temperatura real vs la predicha}
    \label{fig:pariedad_temp_kerne}
\end{figure}

\noindent \textbf{ENTRENANDO MODELO PARA TEMPERATURA}\\
MSE Entrenamiento (Temperatura): 1.1788\\
MSE Validación (Temperatura): 1.4142\\
R\textsuperscript{2} Validación (Temperatura): -1.6287\\

\section{KERNEL RIDGE - Curvas de aprendizaje}

\begin{figure}[H]
    \centering
    \includegraphics[width=0.75\linewidth]{curva de aprendizaje.png}
    \caption{Curvas de aprendizaje de la Humedad y Temperatura}
    \label{fig:curvas_aprendizaje}
\end{figure}

\subsection{Análisis de las curvas de aprendizaje}

\noindent \textbf{HUMEDAD:}\\
- MSE Final Entrenamiento: 4.5065\\
- MSE Final Validación: 7.8341\\
- Diagnóstico: SUBAJUSTE\\

\noindent \textbf{TEMPERATURA:}\\
- MSE Final Entrenamiento: 1.1513\\
- MSE Final Validación: 2.2288\\
- Diagnóstico: SUBAJUSTE\\

\noindent \textbf{INTERPRETACIÓN:}\\
• Si las curvas CONVERGEN: Modelo generaliza bien\\
• Si hay BRECHA grande: Posible sobreajuste\\
• Si ambas son ALTAS: Posible subajuste\\
• Áreas sombreadas: Variabilidad (1 desviación estándar)\\

\section{KERNEL RIDGE - Compromiso Sesgo-Varianza}

\begin{figure}[H]
    \centering
    \includegraphics[width=0.75\linewidth]{descargar.png}
    \caption{Compromiso Sesgo-Varianza de la Temperatura}
    \label{fig:sesgo_varianza_temp}
\end{figure}

\noindent Mejor gamma encontrado: 0.0129\\
MSE validación mínimo: 23.7237\\
MSE entrenamiento correspondiente: 21.4689\\

\begin{figure}[H]
    \centering
    \includegraphics[width=0.75\linewidth]{compromiso ses-variance humedad.png}
    \caption{Compromiso Sesgo-Varianza de la Humedad}
    \label{fig:sesgo_varianza_humedad}
\end{figure}



\noindent \textbf{DIAGNÓSTICO}\\
• Mejor gamma: 0.138950\\
• MSE validación mínimo: 7.1422\\
• MSE entrenamiento correspondiente: 3.6402\\

\noindent \textbf{Diagnóstico del problema:} SOBREAJUSTE\\
El modelo se ajusta demasiado a los datos de entrenamiento\\

\noindent \textbf{Comportamiento de gamma:}\\
• Gamma pequeño (<0.0052): Modelo muy simple (alto sesgo)\\
• Gamma óptimo (~0.1389): Balance ideal\\
• Gamma grande (>19.31): Modelo muy complejo (alta varianza)\\



\section{KERNEL RIDGE - Temperatura}

\begin{figure}[H]
    \centering
    \includegraphics[width=0.5\linewidth]{2.png}
    \caption{CPredicciones del modelo Kernel Ridge}
\end{figure}


\begin{figure}[H]
    \centering
    \includegraphics[width=0.5\linewidth]{3.png}
    \caption{Comparación visual entre valores reales y predichos}
\end{figure}

\begin{figure}[H]
    \centering
    \includegraphics[width=0.5\linewidth]{eeeeeee.png}
    \caption{Distribución de errores del modelo}
\end{figure}

\begin{figure}[H]
    \centering
    \includegraphics[width=0.5\linewidth]{5.png}
    \caption{}
\end{figure}

\begin{table}[H]
    \centering
    \caption{Resultados del modelo Kernel Ridge}
    \begin{tabular}{@{}lc@{}}
        \toprule
        \textbf{Métrica} & \textbf{Kernel Ridge} \\
        \midrule
        MSE (Test) & 0.0330 \\
        R\textsuperscript{2} (Test) & 0.8488 \\
        Error Promedio (°C) & 0.0436 \\
        Error Máximo (°C) & 0.7905 \\
        Error Mínimo (°C) & 0.0003 \\
        Desviación Estándar del Error & 0.0748 \\
        \bottomrule
    \end{tabular}
\end{table}


%%%%%%%%%%%%%%
\section{KERNEL RIDGE - Humedad}

\begin{figure}[H]
    \centering
    \includegraphics[width=0.75\linewidth]{naaaa.png}
    \caption{VISUALIZACIÓN COMPLETA CON TODOS LOS DATOS}
\end{figure}

\begin{figure}[H]
    \centering
    \includegraphics[width=0.75\linewidth]{nddddssd.png}
    \caption{Mejor modelo tomado, regresion lineal, comparacion de humedad predicha vs humedad real}
\end{figure}

\begin{figure}[H]
    \centering
    \includegraphics[width=0.75\linewidth]{xxxx.png}
    \caption{Evolución del error en el tiempo}
\end{figure}



\noindent ESTADÍSTICAS DETALLADAS\\
R² en prueba: 0.9737\\
Error absoluto promedio: 0.0692\%\\
Error máximo: 0.3572\%\\
Error mínimo: 0.0002\%\\
Desviación estándar del error: 0.0625\%\\

\vspace{1em}
\textbf{Comentario/análisis - Kernel Ridge:}
El modelo \textit{Kernel Ridge} logró un ajuste adecuado para la temperatura (R$^2$ = 0.8488), evidenciando buena capacidad predictiva y bajo error promedio. Sin embargo, para la humedad el modelo presentó sobreajuste, lo que indica que aprendió demasiado bien los patrones del conjunto de entrenamiento pero no generalizó correctamente. Esto se observa en la brecha entre el error de entrenamiento y validación. Aun así, es uno de los métodos con mejor desempeño para temperatura dentro del conjunto analizado.

% ======================= REGRESIÓN LINEAL =======================

\section{Regresión Lineal - Temperatura}

% Curva de aprendizaje
\begin{figure}[H]
    \centering
    \includegraphics[width=0.75\linewidth]{img/RLapr.png}
    \caption{Curva de aprendizaje para Temperatura (Regresión Lineal)}
    \label{fig:apr_lineal_temp}
\end{figure}

\noindent \textbf{Análisis de la curva de aprendizaje:}\\
- MSE Final Entrenamiento: 0.1629\\
- MSE Final Validación: 0.1731\\
- Diagnóstico: BUENO\\

- INTERPRETACIÓN:\\
- Si las curvas CONVERGEN: Modelo generaliza bien\\
- Si hay BRECHA grande: Posible sobreajuste\\
- Si ambas son ALTAS: Posible subajuste\\
- Áreas sombreadas: Variabilidad (1 desviación estándar)\\

% Curva de validación (complejidad polinómica)
\begin{figure}[H]
    \centering
    \includegraphics[width=0.75\linewidth]{img/RLval.png}
    \caption{Curva de Validación del Modelo Lineal para Temperatura (variando complejidad polinómica)}
    \label{fig:val_lineal_temp}
\end{figure}

\noindent Mejor grado/configuración: 3\\

% Curva de compromiso sesgo-varianza
\begin{figure}[H]
    \centering
    \includegraphics[width=0.75\linewidth]{img/RLsesgo_varianza.png}
    \caption{Compromiso Sesgo–Varianza - Temperatura (Regresión Lineal Polinómica)}
    \label{fig:sesgo_varianza_lineal_temp}
\end{figure}

\noindent \textbf{Diagnóstico:}\\
• Mejor grado polinómico: 3\\
• MSE validación mínimo: 0.0963\\
• MSE entrenamiento correspondiente: 0.0457\\

Diagnóstico del problema: BALANCE IDEAL\\
El modelo generaliza correctamente

% Curva de paridad mejorada
\begin{figure}[H]
    \centering
    \includegraphics[width=0.5\linewidth]{img/RLpar.png}
    \caption{Curva de paridad de la Temperatura real vs la predicha (Entrenamiento y Validación)}
    \label{fig:paridad_lineal_temp}
\end{figure}

\noindent \textbf{Métricas del modelo lineal:}\\
MSE Entrenamiento: 0.164\\
MSE Validación: 0.166.\\
R\textsuperscript{2} Entrenamiento: 0.667\\
R\textsuperscript{2} Validación: 0.692\\

% Actual vs Predicho (Entrenamiento y Validación)
\begin{figure}[H]
    \centering
    \includegraphics[width=0.75\linewidth]{img/RL_actual_vs_predicho.png}
    \caption{Actual vs Predicho - Regresión Lineal (Temperatura)}
    \label{fig:actual_vs_predicho_lineal_temp}
\end{figure}

% Serie temporal: Temperatura real vs predicha (validación)
\begin{figure}[H]
    \centering
    \includegraphics[width=0.75\linewidth]{img/RLserie_temp.png}
    \caption{Serie temporal de Temperatura real vs predicha (validación)}
    \label{fig:serie_temporal_lineal_temp}
\end{figure}

% Gráfica de errores absolutos en el tiempo
\begin{figure}[H]
    \centering
    \includegraphics[width=0.75\linewidth]{img/RLerror_abs.png}
    \caption{Evolución del error absoluto en el tiempo - Temperatura}
    \label{fig:error_abs_lineal_temp}
\end{figure}

% Histograma de la distribución de errores absolutos
\begin{figure}[H]
    \centering
    \includegraphics[width=0.6\linewidth]{img/RLhist_error.png}
    \caption{Distribución de errores absolutos - Regresión Lineal - Temperatura}
    \label{fig:hist_error_lineal_temp}
\end{figure}

% Tabla resumen de métricas
\begin{table}[H]
    \centering
    \caption{Resultados del modelo Regresión Lineal}
    \begin{tabular}{@{}lc@{}}
        \toprule
        \textbf{Métrica} & \textbf{Regresión Lineal} \\
        \midrule
        MSE (Test) & 0.1657 \\
        R\textsuperscript{2} (Test) & 0.6920 \\
        Error Promedio (°C) & 0.3012 \\
        Error Máximo (°C) & 1.2040 \\
        Error Mínimo (°C) & 0.0036 \\
        Desviación Estándar del Error & 0.2738 \\
        \bottomrule
    \end{tabular}
\end{table}

% ======================= REGRESIÓN LINEAL - HUMEDAD =======================

\section{Regresión Lineal - Humedad}

% Curva de aprendizaje
\begin{figure}[H]
    \centering
    \includegraphics[width=0.75\linewidth]{img/aprendizaje_lineal_hum.png}
    \caption{Curva de aprendizaje para Humedad (Regresión Lineal)}
    \label{fig:apr_lineal_hum}
\end{figure}

\noindent \textbf{Análisis de la curva de aprendizaje:}\\
- MSE Final Entrenamiento: 5.5786\\
- MSE Final Validación: 6.0727\\
- Diagnóstico: SUBAJUSTE\\

% Curva de validación (complejidad polinómica)
\begin{figure}[H]
    \centering
    \includegraphics[width=0.75\linewidth]{img/val_lineal_hum.png}
    \caption{Curva de Validación del Modelo Lineal para Humedad (variando complejidad polinómica)}
    \label{fig:val_lineal_hum}
\end{figure}

\noindent Mejor grado/configuración: 3\\

% Curva de compromiso sesgo-varianza
\begin{figure}[H]
    \centering
    \includegraphics[width=0.75\linewidth]{img/sesgo_varianza_lineal_hum.png}
    \caption{Compromiso Sesgo–Varianza - Humedad (Regresión Lineal Polinómica)}
    \label{fig:sesgo_varianza_lineal_hum}
\end{figure}

\noindent \textbf{Diagnóstico:}\\
• Mejor grado polinómico: 3\\
• MSE validación mínimo: 2.6917\\
• MSE entrenamiento correspondiente: 1.7651\\
\newline Diagnóstico del problema: SUBAJUSTE\\
El modelo es demasiado simple para los datos\\

% Curva de paridad mejorada
\begin{figure}[H]
    \centering
    \includegraphics[width=0.5\linewidth]{img/paridad_lineal_hum.png}
    \caption{Curva de paridad de la Humedad real vs la predicha (Entrenamiento y Validación)}
    \label{fig:paridad_lineal_hum}
\end{figure}

\noindent \textbf{Métricas del modelo lineal:}\\
MSE Entrenamiento: 5.6209\\
MSE Validación: 6.9739\\
R\textsuperscript{2} Entrenamiento: 0.4995\\
R\textsuperscript{2} Validación: 0.4131\\

% Actual vs Predicho (Entrenamiento y Validación)
\begin{figure}[H]
    \centering
    \includegraphics[width=0.75\linewidth]{img/actual_vs_predicho_lineal_hum.png}
    \caption{Actual vs Predicho - Regresión Lineal (Humedad)}
    \label{fig:actual_vs_predicho_lineal_hum}
\end{figure}

% Serie temporal: Humedad real vs predicha (validación)
\begin{figure}[H]
    \centering
    \includegraphics[width=0.75\linewidth]{img/serie_lineal_hum.png}
    \caption{Serie temporal de Humedad real vs predicha (validación)}
    \label{fig:serie_temporal_lineal_hum}
\end{figure}

% Gráfica de errores absolutos en el tiempo
\begin{figure}[H]
    \centering
    \includegraphics[width=0.75\linewidth]{img/error_abs_lineal_hum.png}
    \caption{Evolución del error absoluto en el tiempo - Humedad}
    \label{fig:error_abs_lineal_hum}
\end{figure}

% Histograma de la distribución de errores absolutos
\begin{figure}[H]
    \centering
    \includegraphics[width=0.6\linewidth]{img/hist_error_lineal_hum.png}
    \caption{Distribución de errores absolutos - Regresión Lineal - Humedad}
    \label{fig:hist_error_lineal_hum}
\end{figure}

% Tabla resumen de métricas
\begin{table}[H]
    \centering
    \caption{Resultados del modelo Regresión Lineal (Humedad)}
    \begin{tabular}{@{}lc@{}}
        \toprule
        \textbf{Métrica} & \textbf{Regresión Lineal} \\
        \midrule
        MSE (Test) & 6.9739 \\
        R\textsuperscript{2} (Test) & 0.4131 \\
    Error Promedio & 2.0526\% \\
    Error Máximo & 7.3054\% \\
    Error Mínimo & 0.0278\% \\
    Desviación Estándar del Error & 1.6616\% \\
        \bottomrule
    \end{tabular}
\end{table}
\vspace{1em}
\textbf{Comentario/análisis - Regresión Lineal:}
El modelo lineal tuvo un desempeño sólido en la predicción de temperatura (R$^2$ = 0.6920), mostrando una buena generalización y estabilidad en las métricas. En cambio, para humedad el resultado fue más limitado (R$^2$ = 0.4131), lo que indica que la relación entre las variables podría no ser estrictamente lineal. En conjunto, los resultados muestran que la regresión lineal es eficiente para relaciones directas, pero insuficiente en contextos con mayor complejidad.



% ======================= REGRESIÓN POLINÓMICA =======================

\section{Regresión no Lineal - Temperatura}

% Curva de validación (grado polinómico)
\begin{figure}[H]
    \centering
    \includegraphics[width=0.75\linewidth]{img/val_poly_temp.png}
    \caption{Curva de Validación del Modelo Polinómico para Temperatura}
    \label{fig:val_poly_temp}
\end{figure}

\noindent Mejor grado polinómico: 3\\

% Curva de paridad
\begin{figure}[H]
    \centering
    \includegraphics[width=0.5\linewidth]{img/paridad_poly_temp.png}
    \caption{Curva de paridad de la Temperatura real vs la predicha (Polinómica)}
    \label{fig:paridad_poly_temp}
\end{figure}

\noindent \textbf{ENTRENANDO MODELO POLINÓMICO PARA TEMPERATURA}\\
MSE Entrenamiento: 0.0799\\
MSE Validación: 0.1204\\
R\textsuperscript{2} Validación: 0.7761\\

% Curva de aprendizaje
\begin{figure}[H]
    \centering
    \includegraphics[width=0.75\linewidth]{img/aprendizaje_poly_temp.png}
    \caption{Curva de aprendizaje para Temperatura (Polinómica)}
    \label{fig:apr_poly_temp}
\end{figure}

\noindent \textbf{Análisis de la curva de aprendizaje:}\\
- MSE Final Entrenamiento: 0.1629\\
- MSE Final Validación: 0.1731\\
- Diagnóstico: BUENO\\

% Actual vs Predicho (Entrenamiento y Validación)
\begin{figure}[H]
    \centering
    \includegraphics[width=0.75\linewidth]{img/actual_vs_predicho_poly_temp.png}
    \caption{Actual vs Predicho - Regresión Polinómica (Temperatura)}
    \label{fig:actual_vs_predicho_poly_temp}
\end{figure}

% Serie temporal: Temperatura real vs predicha (validación)
\begin{figure}[H]
    \centering
    \includegraphics[width=0.75\linewidth]{img/serie_poly_temp.png}
    \caption{Serie temporal de Temperatura real vs predicha (validación, Polinómica)}
    \label{fig:serie_temporal_poly_temp}
\end{figure}

% Evolución del error absoluto en el tiempo
\begin{figure}[H]
     \centering
     \includegraphics[width=0.75\linewidth]{img/error_abs_evol_poly_temp.png}
     \caption{Evolución del error absoluto en el tiempo - Regresión Polinómica - Temperatura}
     \label{fig:error_abs_evol_poly_temp}
\end{figure}

% Histograma de la distribución de errores absolutos
\begin{figure}[H]
    \centering
    \includegraphics[width=0.6\linewidth]{img/hist_error_poly_temp.png}
    \caption{Distribución de errores absolutos - Regresión Polinómica - Temperatura}
    \label{fig:hist_error_poly_temp}
\end{figure}


% Tabla resumen de métricas
\begin{table}[H]
    \centering
    \caption{Resultados del modelo Regresión Polinómica (Temperatura)}
    \begin{tabular}{@{}lc@{}}
        \toprule
        \textbf{Métrica} & \textbf{Regresión Polinómica} \\
        \midrule
        MSE (Test) & 0.1204 \\
        R\textsuperscript{2} (Test) & 0.7761 \\
    Error Promedio & 0.2806\% \\
    Error Máximo & 0.7818\% \\
    Error Mínimo & 0.0029\% \\
    Desviación Estándar del Error & 0.2042\% \\
        \bottomrule
    \end{tabular}
\end{table}

% ======================= REGRESIÓN POLINÓMICA - HUMEDAD =======================

\section{Regresión no Lineal- Humedad}

% Curva de validación (grado polinómico)
\begin{figure}[H]
    \centering
    \includegraphics[width=0.75\linewidth]{img/val_poly_hum.png}
    \caption{Curva de Validación del Modelo no Lineal para Humedad}
    \label{fig:val_poly_hum}
\end{figure}

\noindent Mejor grado polinómico: 3\\

% Curva de paridad
\begin{figure}[H]
    \centering
    \includegraphics[width=0.5\linewidth]{img/paridad_poly_hum.png}
    \caption{Curva de paridad de la Humedad real vs la predicha (Regresión no Lineal)}
    \label{fig:paridad_poly_hum}
\end{figure}

\noindent \textbf{ENTRENANDO MODELO POLINÓMICO PARA HUMEDAD}\\
MSE Entrenamiento: 2.2906\\
MSE Validación: 2.8346\\
R\textsuperscript{2} Entrenamiento: 0.7960\\
R\textsuperscript{2} Validación: 0.7615\\

% Curva de aprendizaje
\begin{figure}[H]
    \centering
    \includegraphics[width=0.75\linewidth]{img/aprendizaje_poly_hum.png}
    \caption{Curva de aprendizaje para Humedad (Regresión no Lineal)}
    \label{fig:apr_poly_hum}
\end{figure}

\noindent \textbf{Análisis de la curva de aprendizaje:}\\
- MSE Final Entrenamiento: 2.2468\\
- MSE Final Validación: 2.7875\\
- Diagnóstico: SUBAJUSTE\\
El modelo es demasiado simple para los datos.\\

% Actual vs Predicho (Entrenamiento y Validación)
\begin{figure}[H]
    \centering
    \includegraphics[width=0.75\linewidth]{img/actual_vs_predicho_poly_hum.png}
    \caption{Actual vs Predicho - Regresión no Lineal (Humedad)}
    \label{fig:actual_vs_predicho_poly_hum}
\end{figure}

% Serie temporal: Humedad real vs predicha (validación)
\begin{figure}[H]
    \centering
    \includegraphics[width=0.75\linewidth]{img/serie_poly_hum.png}
    \caption{Serie temporal de Humedad real vs predicha (validación, Regresión no Lineal)}
    \label{fig:serie_temporal_poly_hum}
\end{figure}

% Evolución del error absoluto en el tiempo
\begin{figure}[H]
    \centering
    \includegraphics[width=0.75\linewidth]{img/error_abs_evol_poly_hum.png}
    \caption{Evolución del error absoluto en el tiempo - Regresión no Lineal - Humedad}
    \label{fig:error_abs_evol_poly_hum}
\end{figure}

% Histograma de la distribución de errores absolutos
\begin{figure}[H]
    \centering
    \includegraphics[width=0.6\linewidth]{img/hist_error_poly_hum.png}
    \caption{Distribución de errores absolutos - Regresión no Lineal - Humedad}
    \label{fig:hist_error_poly_hum}
\end{figure}

% Tabla resumen de métricas
\begin{table}[H]
    \centering
    \caption{Resultados del modelo Regresión no Lineal (Humedad)}
    \begin{tabular}{@{}lc@{}}
        \toprule
        \textbf{Métrica} & \textbf{Regresión Polinómica} \\
        \midrule
        MSE (Test) & 2.8346 \\
        R\textsuperscript{2} (Test) & 0.7615 \\
        Error Promedio & 0.2806 \\
        Error Máximo & 0.7818 \\
        Error Mínimo & 0.0029 \\
        Desviación Estándar del Error & 0.2042 \\
        \bottomrule
    \end{tabular}
\end{table}
\vspace{1em}
\textbf{Comentario/análisis - Regresión No Lineal:}
La regresión polinómica (grado 3) mejoró el rendimiento frente al modelo lineal, alcanzando R$^2$ de 0.7761 para temperatura y 0.7615 para humedad. Esto demuestra que la inclusión de términos no lineales permite capturar mejor las variaciones del sistema físico. No obstante, se observó una ligera tendencia al subajuste en los datos de validación, lo que sugiere que el modelo podría beneficiarse de un ajuste más fino del grado polinómico o del número de muestras.


% ======================= RIDGE =======================

\section{Ridge - Temperatura}


% Curva de validación (alpha)
\begin{figure}[H]
    \centering
    \includegraphics[width=0.75\linewidth]{img/val_ridge_temp.png}
    \caption{Curva de Validación del Modelo Ridge para Temperatura (variando $\alpha$)}
    \label{fig:val_ridge_temp}
\end{figure}

\noindent Mejor $\alpha$:  0.01668\\

% Curva de paridad
\begin{figure}[H]
    \centering
    \includegraphics[width=0.5\linewidth]{img/paridad_ridge_temp.png}
    \caption{Curva de paridad de la Temperatura real vs la predicha (Ridge)}
    \label{fig:paridad_ridge_temp}
\end{figure}

\noindent \textbf{Métricas del modelo Ridge:}\\
MSE Entrenamiento: 0.186\\
MSE Validación: 0.208\\
R\textsuperscript{2} Entrenamiento: 0.622\\
R\textsuperscript{2} Validación: 0.613\\

% Curva de aprendizaje
\begin{figure}[H]
    \centering
    \includegraphics[width=0.75\linewidth]{img/aprendizaje_ridge_temp.png}
    \caption{Curva de aprendizaje para Temperatura (Ridge)}
    \label{fig:apr_ridge_temp}
\end{figure}

\noindent \textbf{Análisis de la curva de aprendizaje:}\\
- MSE Final Entrenamiento: 0.3511\\
- MSE Final Validación: 0.3682\\
- Diagnóstico: BALANCE\\

% Actual vs Predicho (Entrenamiento y Validación)
\begin{figure}[H]
    \centering
    \includegraphics[width=0.75\linewidth]{img/ridge_actual_vs_predicho_temp.png}
    \caption{Actual vs Predicho - Ridge (Temperatura)}
    \label{fig:actual_vs_predicho_ridge_temp}
\end{figure}

% Serie temporal: Temperatura real vs predicha (validación)
\begin{figure}[H]
    \centering
    \includegraphics[width=0.75\linewidth]{img/ridge_serie_temp.png}
    \caption{Serie temporal de Temperatura real vs predicha (validación, Ridge)}
    \label{fig:serie_temporal_ridge_temp}
\end{figure}

% Gráfica de errores absolutos en el tiempo
\begin{figure}[H]
    \centering
    \includegraphics[width=0.75\linewidth]{img/ridge_error_abs_temp.png}
    \caption{Evolución del error absoluto en el tiempo - Ridge - Temperatura}
    \label{fig:error_abs_ridge_temp}
\end{figure}

% Histograma de la distribución de errores absolutos
\begin{figure}[H]
    \centering
    \includegraphics[width=0.6\linewidth]{img/ridge_hist_error_temp.png}
    \caption{Distribución de errores absolutos - Ridge - Temperatura}
    \label{fig:hist_error_ridge_temp}
\end{figure}

% Tabla resumen de métricas
\begin{table}[H]
    \centering
    \caption{Resultados del modelo Ridge (Temperatura)}
    \begin{tabular}{@{}lc@{}}
        \toprule
        \textbf{Métrica} & \textbf{Ridge} \\
        \midrule
        MSE (Test) & 0.2083 \\
        R\textsuperscript{2} (Test) & 0.0.6128 \\
        Error Promedio (°C) & 0.3471 \\
        Error Máximo (°C) & 1.2328 \\
        Error Mínimo (°C) & 0.0102 \\
        Desviación Estándar del Error & 0.2964 \\
        \bottomrule
    \end{tabular}
\end{table}


% ======================= RIDGE - HUMEDAD =======================

\section{Ridge - Humedad}


% Curva de validación (alpha)
\begin{figure}[H]
    \centering
    \includegraphics[width=0.75\linewidth]{img/val_ridge_hum.png}
    \caption{Curva de Validación del Modelo Ridge para Humedad (variando $\alpha$)}
    \label{fig:val_ridge_hum}
\end{figure}

\noindent Mejor $\alpha$: 0.05995\\

% Curva de paridad
\begin{figure}[H]
    \centering
    \includegraphics[width=0.5\linewidth]{img/paridad_ridge_hum.png}
    \caption{Curva de paridad de la Humedad real vs la predicha (Ridge)}
    \label{fig:paridad_ridge_hum}
\end{figure}

\noindent \textbf{Métricas del modelo Ridge:}\\
MSE Entrenamiento: 5.743\\
MSE Validación: 7.202\\
R\textsuperscript{2} Entrenamiento: 0.489\\
R\textsuperscript{2} Validación: 0.394\\

% Curva de aprendizaje
\begin{figure}[H]
    \centering
    \includegraphics[width=0.75\linewidth]{img/aprendizaje_ridge_hum.png}
    \caption{Curva de aprendizaje para Humedad (Ridge)}
    \label{fig:apr_ridge_hum}
\end{figure}

\noindent \textbf{Análisis de la curva de aprendizaje:}\\
- MSE Final Entrenamiento: 5.7533\\
- MSE Final Validación: 6.2137\\
- Diagnóstico: SUBAJUSTE\\

% Actual vs Predicho (Entrenamiento y Validación)
\begin{figure}[H]
    \centering
    \includegraphics[width=0.75\linewidth]{img/ridge_actual_vs_predicho_hum.png}
    \caption{Actual vs Predicho - Ridge (Humedad)}
    \label{fig:actual_vs_predicho_ridge_hum}
\end{figure}

% Serie temporal: Humedad real vs predicha (validación)
\begin{figure}[H]
    \centering
    \includegraphics[width=0.75\linewidth]{img/ridge_serie_hum.png}
    \caption{Serie temporal de Humedad real vs predicha (validación, Ridge)}
    \label{fig:serie_temporal_ridge_hum}
\end{figure}

% Gráfica de errores absolutos en el tiempo
\begin{figure}[H]
    \centering
    \includegraphics[width=0.75\linewidth]{img/ridge_error_abs_hum.png}
    \caption{Evolución del error absoluto en el tiempo - Ridge - Humedad}
    \label{fig:error_abs_ridge_hum}
\end{figure}

% Histograma de la distribución de errores absolutos
\begin{figure}[H]
    \centering
    \includegraphics[width=0.6\linewidth]{img/ridge_hist_error_hum.png}
    \caption{Distribución de errores absolutos - Ridge - Humedad}
    \label{fig:hist_error_ridge_hum}
\end{figure}

% Tabla resumen de métricas
\begin{table}[H]
    \centering
    \caption{Resultados del modelo Ridge (Humedad)}
    \begin{tabular}{@{}lc@{}}
        	\toprule
        	\textbf{Métrica} & \textbf{Ridge} \\
        \midrule
        MSE (Test) & 7.2025 \\
        R\textsuperscript{2} (Test) & 0.3939 \\
    Error Promedio & 2.1070\% \\
    Error Máximo & 7.8507\% \\
    Error Mínimo & 0.0955\% \\
    Desviación Estándar del Error & 1.6622\% \\
        \bottomrule
    \end{tabular}
\end{table}
\vspace{1em}
	\textbf{Comentario/análisis - Ridge:}\\
	\textbf{Temperatura:} El modelo Ridge mostró un balance adecuado entre sesgo y varianza, con un R$^2$ de 0.6128. Esto indica que logra generalizar razonablemente bien, manteniendo estabilidad y evitando el sobreajuste.\\
	\textbf{Humedad:} Para humedad, el modelo presentó subajuste (alto sesgo), reflejado en un R$^2$ bajo (0.3939) y errores relativamente altos. Esto sugiere que la capacidad del modelo es insuficiente para capturar la complejidad de la variable humedad.\\
	\textit{En conjunto, Ridge es una opción robusta y estable para temperatura, pero puede resultar demasiado simple para predecir humedad en este caso.}


% ======================= LASSO =======================

\section{Lasso - Temperatura}


% Curva de validación (alpha)
\begin{figure}[H]
    \centering
    \includegraphics[width=0.75\linewidth]{img/val_lasso_temp.png}
    \caption{Curva de Validación del Modelo Lasso para Temperatura (variando $\alpha$)}
    \label{fig:val_lasso_temp}
\end{figure}

\noindent Mejor $\alpha$: 0.0003594\\

% Curva de paridad
\begin{figure}[H]
    \centering
    \includegraphics[width=0.5\linewidth]{img/paridad_lasso_temp.png}
    \caption{Curva de paridad de la Temperatura real vs la predicha (Lasso)}
    \label{fig:paridad_lasso_temp}
\end{figure}

\noindent \textbf{Métricas del modelo Lasso:}\\
MSE Entrenamiento: 0.4694\\
MSE Validación: 0.5199\\
R\textsuperscript{2} Entrenamiento: 0.0456\\
R\textsuperscript{2} Validación: 0.0336\\

% Curva de aprendizaje
\begin{figure}[H]
    \centering
    \includegraphics[width=0.75\linewidth]{img/aprendizaje_lasso_temp.png}
    \caption{Curva de aprendizaje para Temperatura (Lasso)}
    \label{fig:apr_lasso_temp}
\end{figure}

\noindent \textbf{Análisis de la curva de aprendizaje:}\\
- MSE Final Entrenamiento: 0.4654\\
- MSE Final Validación: 0.4781\\
- Diagnóstico: BALANCE\\

% Actual vs Predicho (Entrenamiento y Validación)
\begin{figure}[H]
    \centering
    \includegraphics[width=0.75\linewidth]{img/lasso_actual_vs_predicho_temp.png}
    \caption{Actual vs Predicho - Lasso (Temperatura)}
    \label{fig:actual_vs_predicho_lasso_temp}
\end{figure}

% Serie temporal: Temperatura real vs predicha (validación)
\begin{figure}[H]
    \centering
    \includegraphics[width=0.75\linewidth]{img/lasso_serie_temp.png}
    \caption{Serie temporal de Temperatura real vs predicha (validación, Lasso)}
    \label{fig:serie_temporal_lasso_temp}
\end{figure}

% Gráfica de errores absolutos en el tiempo
\begin{figure}[H]
    \centering
    \includegraphics[width=0.75\linewidth]{img/lasso_error_abs_temp.png}
    \caption{Evolución del error absoluto en el tiempo - Lasso - Temperatura}
    \label{fig:error_abs_lasso_temp}
\end{figure}

% Histograma de la distribución de errores absolutos
\begin{figure}[H]
    \centering
    \includegraphics[width=0.6\linewidth]{img/lasso_hist_error_temp.png}
    \caption{Distribución de errores absolutos - Lasso - Temperatura}
    \label{fig:hist_error_lasso_temp}
\end{figure}

% Tabla resumen de métricas
\begin{table}[H]
    \centering
    \caption{Resultados del modelo Lasso (Temperatura)}
    \begin{tabular}{@{}lc@{}}
        \toprule
        \textbf{Métrica} & \textbf{Lasso} \\
        \midrule
        MSE (Test) & 0.5199 \\
        R\textsuperscript{2} (Test) & 0.0336 \\
        Error Promedio (°C) & 0.5675 \\
        Error Máximo (°C) & 1.9577 \\
        Error Mínimo (°C) & 0.0092 \\
        Desviación Estándar del Error & 0.4448 \\
        \bottomrule
    \end{tabular}
\end{table}


% ======================= LASSO - HUMEDAD =======================

\section{Lasso - Humedad}


% Curva de validación (alpha)
\begin{figure}[H]
    \centering
    \includegraphics[width=0.75\linewidth]{img/val_lasso_hum.png}
    \caption{Curva de Validación del Modelo Lasso para Humedad (variando $\alpha$)}
    \label{fig:val_lasso_hum}
\end{figure}

\noindent Mejor $\alpha$: 0.001292\\

% Curva de paridad
\begin{figure}[H]
    \centering
    \includegraphics[width=0.5\linewidth]{img/paridad_lasso_hum.png}
    \caption{Curva de paridad de la Humedad real vs la predicha (Lasso)}
    \label{fig:paridad_lasso_hum}
\end{figure}

\noindent \textbf{Métricas del modelo Lasso:}\\
MSE Entrenamiento: 8.3500\\
MSE Validación: 10.1422\\
R\textsuperscript{2} Entrenamiento: 0.2564\\
R\textsuperscript{2} Validación: 0.1465\\

% Curva de aprendizaje
\begin{figure}[H]
    \centering
    \includegraphics[width=0.75\linewidth]{img/aprendizaje_lasso_hum.png}
    \caption{Curva de aprendizaje para Humedad (Lasso)}
    \label{fig:apr_lasso_hum}
\end{figure}

\noindent \textbf{Análisis de la curva de aprendizaje:}\\
- MSE Final Entrenamiento: 8.3075\\
- MSE Final Validación: 8.6869\\
- Diagnóstico: SUBAJUSTE\\

% Actual vs Predicho (Entrenamiento y Validación)
\begin{figure}[H]
    \centering
    \includegraphics[width=0.75\linewidth]{img/lasso_actual_vs_predicho_hum.png}
    \caption{Actual vs Predicho - Lasso (Humedad)}
    \label{fig:actual_vs_predicho_lasso_hum}
\end{figure}

% Serie temporal: Humedad real vs predicha (validación)
\begin{figure}[H]
    \centering
    \includegraphics[width=0.75\linewidth]{img/lasso_serie_hum.png}
    \caption{Serie temporal de Humedad real vs predicha (validación, Lasso)}
    \label{fig:serie_temporal_lasso_hum}
\end{figure}

% Gráfica de errores absolutos en el tiempo
\begin{figure}[H]
    \centering
    \includegraphics[width=0.75\linewidth]{img/lasso_error_abs_hum.png}
    \caption{Evolución del error absoluto en el tiempo - Lasso - Humedad}
    \label{fig:error_abs_lasso_hum}
\end{figure}

% Histograma de la distribución de errores absolutos
\begin{figure}[H]
    \centering
    \includegraphics[width=0.6\linewidth]{img/lasso_hist_error_hum.png}
    \caption{Distribución de errores absolutos - Lasso - Humedad}
    \label{fig:hist_error_lasso_hum}
\end{figure}

% Tabla resumen de métricas
\begin{table}[H]
    \centering
    \caption{Resultados del modelo Lasso (Humedad)}
    \begin{tabular}{@{}lc@{}}
        \toprule
        \textbf{Métrica} & \textbf{Lasso} \\
        \midrule
        MSE (Test) & 10.1422 \\
        R\textsuperscript{2} (Test) & 0.1465 \\
    Error Promedio & 2.5323\% \\
    Error Máximo & 9.7014\% \\
    Error Mínimo & 0.1517\% \\
    Desviación Estándar del Error & 1.9312\% \\
        \bottomrule
    \end{tabular}
\end{table}
\vspace{1em}
	\textbf{Comentario/análisis - Lasso:}\\
	\textbf{Temperatura:} Lasso penalizó los coeficientes más débiles, pero el rendimiento fue bajo (R$^2$ ≈ 0.03), indicando que eliminó variables relevantes.\\
	\textbf{Humedad:} Para humedad, el ajuste también fue pobre (R$^2$ ≈ 0.14), reflejando subajuste.\\
	\textit{A pesar de su bajo desempeño, Lasso es útil para identificar variables poco relevantes en el modelo.}


% ======================= ELASTICNET =======================

\section{ElasticNet - Temperatura}

% Curva de validación (alpha/l1_ratio)
\begin{figure}[H]
    \centering
    \includegraphics[width=0.75\linewidth]{img/val_enet_temp.png}
    \caption{Curva de Validación del Modelo ElasticNet para Temperatura (variando $\alpha$ y $l1\_ratio$)}
    \label{fig:val_enet_temp}
\end{figure}

\noindent Mejor $\alpha$: 0.0003594\\

% Curva de paridad
\begin{figure}[H]
    \centering
    \includegraphics[width=0.5\linewidth]{img/paridad_enet_temp.png}
    \caption{Curva de paridad de la Temperatura real vs la predicha (ElasticNet)}
    \label{fig:paridad_enet_temp}
\end{figure}

\noindent \textbf{Métricas del modelo ElasticNet:}\\
MSE Entrenamiento: 0.4452\\
MSE Validación: 0.5067\\
R\textsuperscript{2} Entrenamiento: 0.0949\\
R\textsuperscript{2} Validación: 0.0582\\

% Curva de aprendizaje
\begin{figure}[H]
    \centering
    \includegraphics[width=0.75\linewidth]{img/aprendizaje_enet_temp.png}
    \caption{Curva de aprendizaje para Temperatura (ElasticNet)}
    \label{fig:apr_enet_temp}
\end{figure}

\noindent \textbf{Análisis de la curva de aprendizaje:}\\
- MSE Final Entrenamiento: 0.4431\\
- MSE Final Validación: 0.4612\\
- Diagnóstico: BALANCE\\

% Actual vs Predicho (Entrenamiento y Validación)
\begin{figure}[H]
    \centering
    \includegraphics[width=0.75\linewidth]{img/enet_actual_vs_predicho_temp.png}
    \caption{Actual vs Predicho - ElasticNet (Temperatura)}
    \label{fig:actual_vs_predicho_enet_temp}
\end{figure}

% Serie temporal: Temperatura real vs predicha (validación)
\begin{figure}[H]
    \centering
    \includegraphics[width=0.75\linewidth]{img/enet_serie_temp.png}
    \caption{Serie temporal de Temperatura real vs predicha (validación, ElasticNet)}
    \label{fig:serie_temporal_enet_temp}
\end{figure}

% Gráfica de errores absolutos en el tiempo
\begin{figure}[H]
    \centering
    \includegraphics[width=0.75\linewidth]{img/enet_error_abs_temp.png}
    \caption{Evolución del error absoluto en el tiempo - ElasticNet - Temperatura}
    \label{fig:error_abs_enet_temp}
\end{figure}

% Histograma de la distribución de errores absolutos
\begin{figure}[H]
    \centering
    \includegraphics[width=0.6\linewidth]{img/enet_hist_error_temp.png}
    \caption{Distribución de errores absolutos - ElasticNet - Temperatura}
    \label{fig:hist_error_enet_temp}
\end{figure}

% Tabla resumen de métricas
\begin{table}[H]
    \centering
    \caption{Resultados del modelo ElasticNet (Temperatura)}
    \begin{tabular}{@{}lc@{}}
        \toprule
        \textbf{Métrica} & \textbf{ElasticNet} \\
        \midrule
        MSE (Test) & 0.5067 \\
        R\textsuperscript{2} (Test) & 0.0582 \\
        Error Promedio (°C) & 0.5577 \\
        Error Máximo (°C) & 1.9254 \\
        Error Mínimo (°C) & 0.0006 \\
        Desviación Estándar del Error & 0.4423 \\
        \bottomrule
    \end{tabular}
\end{table}

% ======================= ELASTICNET - HUMEDAD =======================

\section{ElasticNet - Humedad}

% Curva de validación (alpha/l1_ratio)
\begin{figure}[H]
    \centering
    \includegraphics[width=0.75\linewidth]{img/val_enet_hum.png}
    \caption{Curva de Validación del Modelo ElasticNet para Humedad (variando $\alpha$ y $l1\_ratio$)}
    \label{fig:val_enet_hum}
\end{figure}

\noindent Mejor $\alpha$/l1\_ratio: 0.0003594\\

% Curva de paridad
\begin{figure}[H]
    \centering
    \includegraphics[width=0.5\linewidth]{img/paridad_enet_hum.png}
    \caption{Curva de paridad de la Humedad real vs la predicha (ElasticNet)}
    \label{fig:paridad_enet_hum}
\end{figure}

\noindent \textbf{Métricas del modelo ElasticNet:}\\
MSE Entrenamiento: 8.3392\\
MSE Validación: 10.1988\\
R\textsuperscript{2} Entrenamiento: 0.2574\\
R\textsuperscript{2} Validación: 0.1417\\

% Curva de aprendizaje
\begin{figure}[H]
    \centering
    \includegraphics[width=0.75\linewidth]{img/aprendizaje_enet_hum.png}
    \caption{Curva de aprendizaje para Humedad (ElasticNet)}
    \label{fig:apr_enet_hum}
\end{figure}

\noindent \textbf{Análisis de la curva de aprendizaje:}\\
- MSE Final Entrenamiento: 8.2918\\
- MSE Final Validación: 8.7348\\
- Diagnóstico: SUBAJUSTE\\

% Actual vs Predicho (Entrenamiento y Validación)
\begin{figure}[H]
    \centering
    \includegraphics[width=0.75\linewidth]{img/enet_actual_vs_predicho_hum.png}
    \caption{Actual vs Predicho - ElasticNet (Humedad)}
    \label{fig:actual_vs_predicho_enet_hum}
\end{figure}

% Serie temporal: Humedad real vs predicha (validación)
\begin{figure}[H]
    \centering
    \includegraphics[width=0.75\linewidth]{img/enet_serie_hum.png}
    \caption{Serie temporal de Humedad real vs predicha (validación, ElasticNet)}
    \label{fig:serie_temporal_enet_hum}
\end{figure}

% Gráfica de errores absolutos en el tiempo
\begin{figure}[H]
    \centering
    \includegraphics[width=0.75\linewidth]{img/enet_error_abs_hum.png}
    \caption{Evolución del error absoluto en el tiempo - ElasticNet - Humedad}
    \label{fig:error_abs_enet_hum}
\end{figure}

% Histograma de la distribución de errores absolutos
\begin{figure}[H]
    \centering
    \includegraphics[width=0.6\linewidth]{img/enet_hist_error_hum.png}
    \caption{Distribución de errores absolutos - ElasticNet - Humedad}
    \label{fig:hist_error_enet_hum}
\end{figure}

% Tabla resumen de métricas
\begin{table}[H]
    \centering
    \caption{Resultados del modelo ElasticNet (Humedad)}
    \begin{tabular}{@{}lc@{}}
        \toprule
        \textbf{Métrica} & \textbf{ElasticNet} \\
        \midrule
        MSE (Test) & 10.1988 \\
        R\textsuperscript{2} (Test) & 0.1417 \\
    Error Promedio & 2.5301\% \\
    Error Máximo & 9.7765\% \\
    Error Mínimo & 0.1493\% \\
    Desviación Estándar del Error & 1.9487\% \\
        \bottomrule
    \end{tabular}
\end{table}

\vspace{1em}
	\textbf{Comentario/análisis - ElasticNet:}\\
	\textbf{Temperatura:} ElasticNet combinó las ventajas de Ridge y Lasso, pero el rendimiento fue moderado (R$^2$ ≈ 0.06), sin destacar en precisión.\\
	\textbf{Humedad:} Para humedad, el desempeño también fue limitado (R$^2$ ≈ 0.14), mostrando subajuste.\\
	\textit{ElasticNet es útil para mantener la estabilidad del error y evitar el sobreajuste, aunque no fue el más preciso en este caso.}


\vspace{2em}
	\textbf{Conclusión general:}
El análisis comparativo de los modelos de regresión aplicados sobre los datos de temperatura y humedad permitió determinar que las técnicas no lineales y de kernel ofrecen un mejor desempeño predictivo frente a los métodos lineales simples. El modelo \textit{Kernel Ridge} destacó por su alta precisión en la estimación de la temperatura, mientras que la regresión polinómica mostró un equilibrio adecuado entre complejidad y capacidad de generalización. 
En contraste, los modelos \textit{Lasso} y \textit{ElasticNet} presentaron menor desempeño debido a la penalización excesiva de coeficientes relevantes. 
En conjunto, los resultados demuestran la importancia de seleccionar el modelo en función de la naturaleza no lineal de los datos y de validar el ajuste mediante métricas como el R$^2$ y el MSE. Finalmente, el proceso de visualización y diagnóstico confirma la coherencia de los datos obtenidos por el sensor, garantizando la validez del entrenamiento de los modelos propuestos.
